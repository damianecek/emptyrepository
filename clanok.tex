% Metódy inžinierskej práce

\documentclass[10pt,twoside,slovak,coursepaper]{article}

\usepackage[slovak]{babel}
%\usepackage[T1]{fontenc}
\usepackage[IL2]{fontenc} % lepšia sadzba písmena Ľ než v T1
\usepackage[utf8]{inputenc}
\usepackage{graphicx}
\usepackage{url} % príkaz \url na formátovanie URL
\usepackage{hyperref} % odkazy v texte budú aktívne (pri niektorých triedach dokumentov spôsobuje posun textu)

\usepackage{cite}
%\usepackage{times}

\pagestyle{headings}

\title{Semestrálny projekt v predmete Metódy inžinierskej práce, ak. rok 2022/23, vedenie: M. Y. Momand} % meno a priezvisko vyučujúceho na cvičeniach

\author{Damián Parigal\\[2pt]
	{\small Slovenská technická univerzita v Bratislave}\\
	{\small Fakulta informatiky a informačných technológií}\\
	{\small \texttt{xparigal@stuba.sk}}
	}

\date{\small 6. október 2022} % upravte



\begin{document}

\maketitle

\begin{abstract}

Virtuálna realita, počítačové, konzolové a mobilné hry už pár rokov ovládajú svet. Preto by sa ani odvetvie vzdelávania nemalo od tejto témy dištancovať alebo hádzať na tento spôsob výučby tieň. Deti sa už od útleho veku najviac učia hrou, či už ide o rozoznávanie farieb, písmen alebo učenie sa nových slov prípadne cudzí jazyk. Správne spracovanými a navrhnutými hrami by sme sa vedeli naučiť omnoho viac ak by sme túto problematiku brali s vážnosťou. Rôzne herné prvky spojené so simuláciami alebo grafickými ukazovazeľmi by sa dali uplatniť v rôznych oblastiach výučby a odboroch ako fyzika, chémia, informatika a omnoho viac.

\end{abstract}



\section{Úvod}

Počítačové hry majú veľmi pozitívny vplyv na zdokonaľovanie pozornosti študentov. V tomto projekte by som sa chcel zamerať na konkrétne riešenie ubúdajucej pozornosti študentov od vzdelávania prostredníctvom počítačových hier.

\section{Nejaká časť} \label{nejaka}




\section{Iná časť} \label{ina}




\section{Dôležitá časť} \label{dolezita}




\section{Ešte dôležitejšia časť} \label{dolezitejsia}




\section{Záver} \label{zaver} % prípadne iný variant názvu



%\acknowledgement{Ak niekomu chcete poďakovať\ldots}


% týmto sa generuje zoznam literatúry z obsahu súboru literatura.bib podľa toho, na čo sa v článku odkazujete
\bibliography{ref.bib}
\bibliographystyle{plain} % prípadne alpha, abbrv alebo hociktorý iný
\end{document}
