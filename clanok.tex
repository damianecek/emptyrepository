% Metódy inžinierskej práce

\documentclass[10pt,twoside,slovak,coursepaper]{article}

\usepackage[slovak]{babel}
%\usepackage[T1]{fontenc}
\usepackage[IL2]{fontenc} % lepšia sadzba písmena Ľ než v T1
\usepackage[utf8]{inputenc}
\usepackage{graphicx}
\usepackage{url} % príkaz \url na formátovanie URL
\usepackage{hyperref} % odkazy v texte budú aktívne (pri niektorých triedach dokumentov spôsobuje posun textu)

\usepackage{cite}
%\usepackage{times}

\pagestyle{headings}

\title{Semestrálny projekt v predmete Metódy inžinierskej práce, ak. rok 2022/23, vedenie: M. Y. Momand} % meno a priezvisko vyučujúceho na cvičeniach

\author{Damián Parigal\\[2pt]
	{\small Slovenská technická univerzita v Bratislave}\\
	{\small Fakulta informatiky a informačných technológií}\\
	{\small \texttt{xparigal@stuba.sk}}
	}

\date{\small 6. október 2022} % upravte



\begin{document}

\maketitle

\begin{abstract}
\ldots
\end{abstract}



\section{Úvod}



\section{Nejaká časť} \label{nejaka}




\section{Iná časť} \label{ina}




\section{Dôležitá časť} \label{dolezita}




\section{Ešte dôležitejšia časť} \label{dolezitejsia}




\section{Záver} \label{zaver} % prípadne iný variant názvu



%\acknowledgement{Ak niekomu chcete poďakovať\ldots}


% týmto sa generuje zoznam literatúry z obsahu súboru literatura.bib podľa toho, na čo sa v článku odkazujete
\bibliography{literatura.bib}
\bibliographystyle{plain} % prípadne alpha, abbrv alebo hociktorý iný
\end{document}
